\usepackage{ifthen}

%valos komplex
\newcommand*{\R}{\ensuremath{\mathsf{R}}}
\newcommand*{\C}{\ensuremath{\mathsf{C}}}

%C ad m times n%
\newcommand*{\Cnm}[2]{\ensuremath{\C^{#1 \times #2}}}
\newcommand*{\Rnm}[2]{\ensuremath{\mathbb{R}^{#1 \times #2}}}

%normak
% #1 (opc.) a norma tipusa, #2 aminek a normajat nezzuk
\newcommand{\norma}[2][nincs]{\ensuremath{
        \ifthenelse{\equal{#1}{nincs}}{
                {\left\| #2 \right\| {} }
        }{
                {\left\| #2 \right\|_{#1} {}}
        }
}}
% ||.|| a norma tipusa opc.
\newcommand{\normafv}[1][nincs]{\norma[#1]{\cdot}}

%lekepezes f: x->y, ahol #1 f (opc), #2 x �s #3 y
\newcommand{\lekep}[3][nincs]{\ensuremath{
        \ifthenelse{\equal{#1}{nincs}}{
                #2 \to #3
        }{
                #1 \colon #2 \to #3
        }
}}

%halmaz {x| f(x)>0} pelda: \halmaz{x}{f(x)>0}
\newcommand{\halmaz}[2]{\ensuremath{
        \{\,#1 \mid #2\,\}
}}
%\max ala elso sorba #1, masodik sorba #2 kerul
\newcommand{\maxcc}[2]{\max_{\substack{{#1} \\ {#2}}} }
\newcommand{\mincc}[2]{\min_{\substack{{#1} \\ {#2}}} }


%logika
\newcommand{\kov}{\Longrightarrow}
\newcommand{\ekv}{\Longleftrightarrow}

%relacios jel
\newcommand{\defegy}{\stackrel{\mathrm{def}}{=}}
\newcommand{\legy}{:=}

%fuggvenyek
\newcommand{\traceCikk}{\mathop{\textup{tr}}}
%\newcommand{\nyomCikk}{\mathop{\textrm{nyom}}}
\newcommand{\rankCikk}{\mathop{\textup{rk}}}
\newcommand{\diagCikk}{\mathop{\textup{diag}}}
\newcommand{\nyom}[1]{\traceCikk(#1)}
\newcommand{\rang}[1]{\rankCikk(#1)}
\newcommand{\range}[1]{\mathop{\textup{Im}}(#1)}
\newcommand{\nullter}[1]{\mathop{\textup{Ker}}(#1)}
\newcommand{\spanja}[1]{ \mathop{\textup{span}} \{ #1 \} }
%pszeudoinverz
\newcommand{\pinv}[1]{{#1}^\dag}

% Az M matrix i-ik oszlopa illetve sora #1=M , #2=i 
\newcommand{\oszl}[2]{{#1}_{(#2)}}
\newcommand{\sor}[2]{{#1}^{(#2)}}
\newcommand{\diag}[1]{\diagCikk \langle #1 \rangle}
\newcommand{\argmax}{\mathop{\textup{argmax}}\limits}
\newcommand{\dist}{{\mathop{\textup{dist}}}}
\renewenvironment{matrix}[1]{ \left( \begin{array}{#1} }{ \end{array} \right) }
%#1 egy oszlopvektor, #2 az oszlop indexe
\newcommand{\oelem}[2]{[#1]_{#2}}
\newcommand{\sorIkn}{I^{n,k}}
\newcommand{\oszlImk}{I_{m,k}}

%infix operatorok
\newcommand{\tart}{\longrightarrow}

%szunetek
\newcommand{\aholszunet}{\qquad}

%betutipus
%linearis ter jelolesere hasznalt betutipus
\newcommand{\ter}[1]{\ensuremath{{\cal #1}}}








%%% Local Variables: 
%%% mode: latex
%%% TeX-master: "matmakro"
%%% End: 

