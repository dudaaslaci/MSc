\usepackage{graphics,latexsym}
\usepackage{ifthen}
\frenchspacing

\def\szamozasMelyseg{section}

%%%%%%Tetelszeru kornyezetek%%%%

\newtheorem{tetel}{t�tel}[\szamozasMelyseg]
\newtheorem{allitas}[tetel]{�ll�t�s}
\newtheorem{defin}[tetel]{defin�ci�}
\newtheorem{lemma}[tetel]{lemma}
\newtheorem{megj}[tetel]{megjegyz�s}
\newtheorem{sejtes}[tetel]{sejt�s}
\newtheorem{kovetkezmeny}[tetel]{k�vetkezm�ny}

\newcommand\van[1]{\not \equal{#1}{nincs}}


        

%%%%A biz k�rnyezet%%%%
%/qed
\def\sqr#1#2{{\vcenter{\vbox{\hrule height .#2pt \hbox{\vrule width .#2pt height#1pt \kern#1pt \vrule width .#2pt} \hrule height .#2pt}}}}
\def\square{\sqr56~}
\def\qed{\hfill $\square$ \bigskip \\}

\newcommand{\bizstilus}[1]{\noindent{\bf #1 \ \ }}
 
% #1 a bizonyitas forrasa, a parameter opcionalis 
\newenvironment{biz}[1][nincs]{
        \ifthenelse{\equal{#1}{nincs}}{
                \bizstilus{Bizony�t�s:}
        }{
                \bizstilus{Bizony�t�s (#1):}
        }
}{\qed}

% biz* #2 kotelezo arg. a bizonyitas sorszama, #1 opc parameter a biz. forrasa
\newenvironment{biz*}[2][nincs]{
        \ifthenelse{\equal{#1}{nincs}}{
                \bizstilus{#2.~bizony�t�s:}
        }{
                \bizstilus{#2.~bizony�t�s (#1):}
        }
}{\qed}

% \lasd(Stewart,66-82.oldal) #1 oldal (opc), #2 hivatkozas
\newcommand{\lasd}[2][nincs]{{
        \ifthenelse{\not \equal{#1}{nincs}}{
                l�sd: #2, #1.~oldal}{
                l�sd: #2}}}
\newcommand{\Lasd}[2][nincs]{
        \ifthenelse{\not \equal{#1}{nincs}}{
                l�sd: #2, #1.~oldal}{
                l�sd: #2}}

%betu tipus
%\emph{} uj fogalomnal
%\idez{...}, ahol a fogalom pontatlan, csak szemleletet ad, vagy hangsulyos
\newcommand{\idez}[1]{,,#1''}

%%% Local Variables: 
%%% mode: latex
%%% TeX-master: t
%%% End: 
